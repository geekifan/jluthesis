\chapter{引论}
\label{chap:introduction}

本章为吉林大学本科生毕业论文 \LaTeX 模板的示例文档\footnote{本模板适配了吉林大学计算机科学
与技术学院的毕业论文格式,参考了《附件Z-3:学院本科毕业论文(设计)撰写参考模板》中的格式要求,
适配作者为geekifan,参见 GitHub 主页\url{https://github.com/geekifan}。},对写作过程
中常见用法和问题进行介绍说明。第一章的内容参考了北京大学硕士学位论文模板\footnote{\url{https://github.com/iofu728/pkuthss}}
的相关章节。

\section{封面及配置文件相关}
\label{sec:cover-pkuthssinfo}

本模板使用 \verb|jluthesis.cfg| 配置文件对封面的相关信息进行设置。示例配置如下:

\begin{Verbatim}
    \ProvidesFile{jluthesis.cfg}
    \def\cover@thesis{毕业设计(论文)}
    \def\cover@ctitlef{第一行}
    \def\cover@ctitles{第二行}
    \def\cover@ctitle{中文完整题目}
    \def\cover@etitlef{The first}
    \def\cover@etitles{The second}
    \def\cover@etitle{English Title}
    \def\cover@school{某某学院}
    \def\cover@major{某某专业}
    \def\cover@author{张三}
    \def\cover@eauthor{Zhang San}
    \def\cover@studentid{11111111}
    \def\cover@mentor{李四}
    \def\cover@ementor{Li Si}
    \def\cover@time{2023 年 6 月}
    \def\abstract@cabstract{摘\hspace{1em}要}
    \def\abstract@eabstract{Abstract}
    \def\abstract@ckeywords{关键词:}
    \def\abstract@ekeywords{Keywords:}
    \def\commitment@title{吉林大学学士学位论文(设计)承诺书}
    \def\commitment@content{本人郑重承诺:所呈交的学士学位毕业论文(设计),
    是本人在指导教师的指导下,独立进行实验、设计、调研等工作基础上取得的成果。
    除文中已经注明引用的内容外,本论文(设计)不包含任何其他个人或集体
    已经发表或撰写的作品成果。对本人实验或设计中做出重要贡献的个人或集体,
    均已在文中以明确的方式注明。本人完全意识到本承诺书的法律结果由本人承担。}
    \def\commitment@sign{承诺人:}
    \def\commitment@time{2023年5月29日}
    \endinput
\end{Verbatim}


\section{字体设定}
\label{sec:fontset}

本模板的默认中文字体为宋体,英文字体为 Times New Roman。本项目不包含任何所需的字体文件。如需使用 Overleaf,请自行在项目中上传 Overleaf 缺失的宋体文件,然后在 \verb|jluthesis.cls| 文件中将 
\begin{Verbatim}
\setCJKfamilyfont{cusong}[AutoFakeBold = {2.17}]{SimSun}
\end{Verbatim}
修改为
\begin{Verbatim}
\setCJKfamilyfont{cusong}[AutoFakeBold = {2.17}]{/path/to/simsun.ttf}
\end{Verbatim}

\section{承诺书}
\label{sec:commitment}

本模板自带承诺书。请核对该模板使用的承诺书与学院使用的承诺书是否一致。如需去除承诺书一页。请将\verb|main.tex|中的\verb|\commitment|注释或删除。

\section{摘要部分}
\label{sec:abstract}

请在\verb|cabstract.tex|中书写中文摘要,在\verb|eabstract.tex|中书写英文摘要。


\section{图表相关}
\label{sec:table-figure}

\subsection{表格样例}
\label{sec:table-example}

一般学术论文使用三线表(如表~\ref{tab:example-table-basic}),需要依赖宏包\verb|booktabs|,使用\verb|\toprule|,\verb|\midrule|,\verb|\bottomrule|控制三线。
此外表序和表名位于表格的上方。
如果需要对表格内进行脚注,可通过\texttt{minipage}中嵌套\texttt{tabular}来实现,具体可参考Stack Overflow\footnote{\url{https://stackoverflow.com/questions/2888817/footnotes-for-tables-in-latex}}。

{\begin{longtable}[c]{c*{7}{r}}
\caption[续表]{续表样例表。}
\label{tab:example-table-continue}\\
\toprule[1.5pt]
 \multicolumn{1}{c}{年龄} & 性别 & \multicolumn{1}{c}{cp} & \multicolumn{1}{c}{静息血压} & \multicolumn{1}{c}{chol}
& \multicolumn{1}{c}{空腹血糖>} & \multicolumn{1}{c}{restecg} & \multicolumn{1}{c}{thalachh} \\
\multicolumn{1}{c}{(岁)} & & \multicolumn{1}{c}{胸痛型}&
\multicolumn{1}{c}{毫米汞柱}& \multicolumn{1}{c}{胆固醇}& \multicolumn{1}{c}{
   120 mg/dl}& 静息状态 & 最大心率 \\\midrule[1pt]
\endfirsthead
\multicolumn{8}{c}{续表~\thetable\hskip1em 续表样例表。}\\
\toprule[1.5pt]
 \multicolumn{1}{c}{年龄} & 性别 & \multicolumn{1}{c}{cp} & \multicolumn{1}{c}{静息血压} & \multicolumn{1}{c}{chol}
& \multicolumn{1}{c}{空腹血糖>} & \multicolumn{1}{c}{restecg} & \multicolumn{1}{c}{thalachh} \\
\multicolumn{1}{c}{(岁)} & & \multicolumn{1}{c}{胸痛型}&
\multicolumn{1}{c}{毫米汞柱}& \multicolumn{1}{c}{胆固醇}& \multicolumn{1}{c}{
   120 mg/dl}& 静息状态 & 最大心率 \\\midrule[1pt]
\endhead
\hline
\multicolumn{8}{r}{续下页}
\endfoot
\endlastfoot
63 & 1 & 3 & 145 & 233 & 1 & 0 & 150 \\
37 & 1 & 2 & 130 & 250 & 0 & 1 & 187 \\
41 & 0 & 1 & 130 & 204 & 0 & 0 & 172 \\
56 & 1 & 1 & 120 & 236 & 0 & 1 & 178 \\
57 & 0 & 0 & 120 & 354 & 0 & 1 & 163 \\
57 & 1 & 0 & 140 & 192 & 0 & 1 & 148 \\
56 & 0 & 1 & 140 & 294 & 0 & 0 & 153 \\
44 & 1 & 1 & 120 & 263 & 0 & 1 & 173 \\
52 & 1 & 2 & 172 & 199 & 1 & 1 & 162 \\
57 & 1 & 2 & 150 & 168 & 0 & 1 & 174 \\
54 & 1 & 0 & 140 & 239 & 0 & 1 & 160 \\
48 & 0 & 2 & 130 & 275 & 0 & 1 & 139 \\
49 & 1 & 1 & 130 & 266 & 0 & 1 & 171 \\
64 & 1 & 3 & 110 & 211 & 0 & 0 & 144 \\
\bottomrule[1.5pt]
\end{longtable}
\footnotesize 注:数据来源于Kaggle Heart Attack Analysis \& Prediction Data Set。}

如需要注明表格中数据来源,则可使用类似的方式,见表~\ref{tab:example-table-source-foot}。

\begin{table*}[htb]
    \centering
    \begin{minipage}[t]{0.55\linewidth} %
        \caption[表格脚注样例表]{表格脚注样例表。表名可通过中括号添加缩略名。}
        \label{tab:example-table-basic}
        \begin{small}
        \begin{tabular}{@{}lccccc@{}}
         \toprule[1.5pt]
         & \textbf{X} & \textbf{Y} & \textbf{Z} & \textbf{N} & \textbf{M} \\
         \midrule[1pt]
            默认        & 99.99 & 99.99 & 99.99 & 99.99\footnote{表格中的脚注1} & 99.99 \\
          \quad w/o X   & 99.99 & 99.99 & 99.99 & 99.99 & 99.99 \\
          \quad w/o Y   & 99.99 & 99.99 & 99.99 & 99.99 & 99.99 \\
          \quad w/o Z   & 99.99\footnote{表格中的脚注2} & 99.99 & 99.99 & 99.99 & 99.99 \\
          \quad w/o N   & 99.99 & 99.99 & 99.99 & 99.99 & 99.99 \\
          \quad w/o M   & 99.99 & 99.99 & 99.99 & 99.99 & 99.99 \\
          \bottomrule[1.5pt]
        \end{tabular}
        \end{small}
    \end{minipage}
\end{table*}

\begin{table*}[htbp]
   \centering
   \caption[数据来源注释表]{表格数据来源注释样例表。}
   \label{tab:example-table-source-foot}
   \begin{minipage}[t]{0.9\textwidth}
   \begin{small}
   \begin{tabular}{@{}l|ccc|ccc@{}}
   \toprule
   \multirow{2}{*}{\textbf{Model}} & \multicolumn{3}{c|}{\textbf{数据集A}} & \multicolumn{3}{c}{\textbf{数据集B}} \\ \cmidrule(l){2-7} 
    & \textbf{指标a}(\%) & \textbf{指标b}(\%) & \textbf{指标c} & \textbf{指标a} (\%) & \textbf{指标b}(\%) & \textbf{指标c} \\ \midrule
    Devlin et al.\cite{devlin2018bert}      &99.99  & 99.99  & 99.99  &99.99  & 99.99  & 99.99  \\
    Yang et al.\cite{yang2019xlnet}      &99.99  & 99.99  & 99.99  &99.99  & 99.99  & 99.99  \\
    \bottomrule
   \end{tabular}\\[6pt]
   \footnotesize 注:数据来源XXXXXX。\\
   \end{small}
   \end{minipage}
\end{table*}

当表格较大,不能在一页内打印时,可以“续表”的形式另页打印,可使用宏包\verb|longtable|实现,如表~\ref{tab:example-table-continue}。


\subsection{图片样例}
\label{sec:figure-example}

\begin{figure}[htb]
  \centering
  \subfloat[吉林大学校徽]{
    \label{sfig:example-fig-logo-fig}
    \includegraphics[height=2.5cm]{figures/jlu-fig-logo}}\hspace{4em}
  \subfloat[吉林大学中文校名]{
    \label{sfig:example-fig-logo-text}
    \includegraphics[height=2.5cm]{figures/jlu-text-logo}}
  \caption{包含子图形的大图形}
  \label{fig:example-fig-subfloat}
\end{figure}

当需要插入多个子图的时候,可以选用宏包\verb|subfloat|,不推荐使用
\verb|subfigure| 和 \verb|subtable|。

若使用继承于\verb|subfigure|的宏包,例如\verb|subfloat|、\verb|subfigure|等,则可直接使用引用\verb|\ref{sfig:xxxx}|引用子图label,如图~\ref{sfig:example-fig-logo-fig}。
否则需要引用主图,再单独标注子图序号,以便符合学位论文要求。

此外,与表格相反,图序和图名需要位于图片的下方。
如果含有子图,每个子图需要具有相应的子图名。


如果需要并排使用两个独立的图形,分别编排图序,则可使用\verb|minipage|,如图~\ref{fig:example-fig-abreast-1}和图~\ref{fig:example-fig-abreast-2}。

\begin{figure}[htb]
\begin{minipage}{0.48\textwidth}
  \centering
  \includegraphics[height=2cm]{figures/jlu-fig-logo}
  \caption{吉林大学校徽}
  \label{fig:example-fig-abreast-1}
\end{minipage}\hfill
\begin{minipage}{0.48\textwidth}
  \centering
  \includegraphics[height=2cm]{figures/jlu-text-logo}
  \caption{吉林大学中文校名}
  \label{fig:example-fig-abreast-2}
\end{minipage}
\end{figure}

\section{公式}
\label{sec:equation}

公式使用通用\LaTeX{}规范即可。对于复杂公式需求,可使用\verb|amsmath|宏包结合Mathpix\footnote{\url{https://mathpix.com/}}等自动化识别工具。

\begin{multline*}
\int_a^b\biggl\{\int_a^b[f(x)^2g(y)^2+f(y)^2g(x)^2]
 -2f(x)g(x)f(y)g(y)\,dx\biggr\}\,dy \\
 =\int_a^b\biggl\{g(y)^2\int_a^bf^2+f(y)^2
  \int_a^b g^2-2f(y)g(y)\int_a^b fg\biggr\}\,dy
\end{multline*}

上述公式来源于刘宝碇的《不确定规划》\cite{liu2003uncertain}。

\section{参考文献}
\label{sec:bibtex}

参考文献使用\verb|gb7714-2015|bibstyle进行管理,具体引用命令与日常使用类似,
例如\verb|\cite{}|。使用本模板之前,请先按照该仓库\footnote{\url{https://github.com/Haixing-Hu/GBT7714-2005-BibTeX-Style}},
将相关bibstyle进行安装,或将其bst文件放入项目目录下直接使用。
